\documentclass[11pt,a4paper]{scrreprt}
\pdfminorversion=7
\usepackage[T1]{fontenc}
\usepackage[ngerman]{babel}
\usepackage{scrhack}
\usepackage{amsmath}
\usepackage{amsfonts}
\usepackage{amssymb}
\usepackage[]{siunitx}  %% For proper units (e.g. \si{kg.m.s^{-1}}) and numbers 
%%(e.g. \num{.3e45})
\sisetup{output-decimal-marker = {,}}
\usepackage{icomma} %% entfernt das Leerzeichen hinter einem Komma in der 
%%Matheumgebung bei 3,141, aber nicht bei f(x, y)
\usepackage[pdftex]{graphicx}
\usepackage{epstopdf}
\usepackage[hidelinks]{hyperref}
\usepackage{float}
\usepackage{placeins}%% Ergmöglich \FloatBarrier
\usepackage[labelfont=bf]{caption}
\captionsetup[table]{justification=raggedright,singlelinecheck=false}
\usepackage{url}   
%% For citing webpages
\usepackage{csquotes} %% \enquote{Text} \foreignquote{Sprache}{Text}
\usepackage[backend=biber, %% Hilfsprogramm "biber" (statt "biblatex" oder "bibtex")
style=numeric, %% Zitierstil (siehe Dokumentation)
sorting=none, %% none => choronoligisch
natbib=true, %% Bereitstellen von natbib-kompatiblen Zitierkommandos
hyperref=true, %% hyperref-Paket verwenden, um Links zu erstellen
]{biblatex}
\addbibresource{Kapitel/x_literaturverzeichnis.bib}
\usepackage{acronym}
\usepackage{tabularx}
\usepackage{multirow}
\usepackage{hhline}
\usepackage{subcaption}
\usepackage{multicol} % Multi Columns für enumerations
%%%%%%%%%%%%%%%%%% Document Style
\hfuzz=2pt %% Warnung für Bad Box erst ab 2pt
\usepackage{lmodern}		%% Spezial Font
\usepackage{xcolor} 	%% farbelicher text
\usepackage[headsepline=true]{scrlayer-scrpage}
\pagestyle{scrheadings}
\lohead[]{\leftmark}	% bei odd page \lohead,\cohead und \rohead
\cohead[]{}				% bei even page \lehead,\cehead und \rehead
%\rohead[]{\rightmark} 	% bei zweiseitig \ihead, \chead \ohead
\rohead[]{{\small FlexHyX - Projekt}}
\automark[section]{chapter}
%%%%%%%%%%%%%%%%% Hurenkind und Schusterjunge
\clubpenalty=10000
\widowpenalty=10000 
\displaywidowpenalty=10000
%%%%%%%%%%%%%%%%%
%\usepackage{blindtext}
\author{Tobias Held}
\title{Erstellung eines Simulationsmodells zur Ermittlung von Lastszenarien in 
	der Binnenschifffahrt}



\begin{document}
	\begin{titlepage}
	%\setlength{\parindent}{0pt}%Einrückung auf Titelseite verhindern
	\begin{figure}[htbp]
		\includegraphics[height=1.4cm]{Kapitel/xx_Logo_HBRS_74mm_Pfade-eps-converted-to.pdf}
	%	\hfill
	%	
	%\includegraphics[height=1.4cm]{Kapitel/xx_BRS-Blau_Schwarz_Ohne_Hintergrund_HD.png}
	\end{figure}
 % \includegraphics[width=8cm]{Logo_HBRS_74mm_Pfade.eps}
  \linespread{1.4}%\renewcommand{\baselinestretch}{1.4}\normalsize
  \vspace{2cm}
  \begin{center}

%% einen Typ auswählen
    \begin{Huge}\textbf{Masterprojekt 2}\end{Huge} \\
    \vspace{1cm}
%% einen Studiengang auswählen
    \begin{Large}{Virtuelle Produktentwicklung}\end{Large} \\
    \begin{large}
    	Forschungsprojekt FlexHyX
    \end{large}

    \vspace{0.7cm}
    \linespread{1.2}%\renewcommand{\baselinestretch}{1.2}\normalsize
    \begin{huge}
      \textbf{???\\}
    \end{huge}
    \linespread{1.5}%\renewcommand{\baselinestretch}{1.5}
    \normalsize
    \vspace{1cm}%{0.7cm}

    \begin{Large}{Tobias Held \\}
    \end{Large}

    \begin{Large}% Fachbereich
        Fachbereich Elektrotechnik, Maschinenbau \\ und Technikjournalismus
    \end{Large}
  \end{center}

  \vspace{3.0cm}

  \begin{large}
    {
      \begin{tabular}{ll}
      Erstprüferin:  & Prof. Dr. Tanja Clees \\
      Zweitprüfer: & Prof. Dr. Gerd Steinebach \\
      Eingereicht am: & \today \\%\today \\ %% or type in: 01. Januar 2017
      Matrikelnummer: & 9028598
      \end{tabular}
	}
  \end{large}

\end{titlepage}
	%\pagestyle{empty}
\section*{Erklärung}

Tobias Held\\
Adresse\\

„Ich versichere hiermit, die von mir vorgelegte Arbeit selbstständig verfasst zu haben. Alle Stellen, die wörtlich oder sinngemäß aus veröffentlichten oder nicht veröffentlichten Arbeiten anderer entnommen sind, habe ich als entnommen kenntlich gemacht. Sämtliche Quellen und Hilfsmittel, die ich für die Arbeit benutzt habe, sind angegeben. Die Arbeit hat mit gleichem Inhalt bzw. in wesentlichen Teilen noch keiner anderen Prüfungsbehörde vorgelegen.

Mir ist bewusst, dass sich die Hochschule vorbehält, meine Arbeit auf plagiierte Inhalte hin zu überprüfen und dass das Auffinden von plagiierten Inhalten zur Nichtigkeit der Arbeit, zur Aberkennung des Abschlusses und zur Exmatrikulation führen können.“
\vspace{3cm}


\noindent\parbox[t]{5cm}{\underline{\hspace{5cm}}\\\noindent Ort, Datum}%
\hfill%
\noindent\parbox[t]{5cm}{\noindent\underline{\hspace{5cm}}\\\noindent Unterschrift}%

	\pagenumbering{Roman}
	\tableofcontents
	\listoffigures		%% Abbildungsverzeichnis || List of Figures
	\listoftables
	\newpage
\chapter*{Abkürzungsverzeichnis}
\begin{acronym}
%	\acro{hm}[H\&M]{Holtrop \& Mennen}
\end{acronym}

\newpage
% Hier evtl: https://www.ctan.org/pkg/glossaries
\chapter*{Variablenverzeichnis}
\begin{table}[ht]
	\begin{tabular}{rl}
	
	\(\beta\) & Trimm: \(m\)\\		
	\(\rho\)& Wasserdichte: \(kg\cdot m^{-3}\)	\\		
	&	\\
	\(A_b\)& Blockfläche Schiff: \(m^2\)\\
	\(B_c\)& Normierte Breite: \(m\)\\
	\(Fn\) & Froude-Zahl: \(-\)\\
	\(Fn_h\) & Froude'sche Tiefenzahl: \(-\)\\	
	\(g\) & Erdbeschleunigung: \(m \cdot s^{-2}\)\\
	\(h\)& Wassertiefe: \(m\)\\
	\(H_m\)& Normierte Wassertiefe: \(m\)\\
	\(lcb\) & Schwerpunkt der Verdrängung: \(\%\)\\
	\(P_b\)& Benässte Schiffsfläche im Querschnitt: \(m\)\\
	\(P_c\)& Benässte Flussfläche im Querschnitt: \(m\)\\
	\(R\) & Widerstand: \(kN\)\\
	\(R_h\)& Hydraulischer Radius Schiff-Wasserstraße: \(m\)\\		
	
	\(u\) & Rückströmung: \(m\cdot s^{-1}\)\\
	\(W\)& Breite Wasseroberfläche: \(m\)\\
	\(w\)& Breite Flussbett: \(m\)\\
	\(S_d\) & Squat: \(m\)\\	
 	\(z\) & Absenkung des Wasserstands: \(m\)\\
\end{tabular}
\end{table}
	\newpage
	\pagenumbering{arabic}
	\pagestyle{scrheadings}
	
	\chapter{Einleitung}\label{ch:Einleitung}
Die Ladung eines Binnenschiffs entlastet den Straßenverkehr um 40 bis 180 LKW. 
%{2019-12-02_STG2018-Dahlke-Friedhoff_Energietraeger_fuer_die_Binnenschifffahrt_von_Morgen}
Im Vergleich zu einem 40-Tonnen-Sattelzug verbraucht ein Schiff weniger 
als ein Drittel an Energie und stößt entsprechend weniger Treibhausgase aus. 
In der Luftschadstoffbilanz ist der Transport per Binnenschiff jedoch 
schlechter als per Lkw.\cite{UmweltBundesamt.2020}
%https://www.umweltbundesamt.de/themen/verkehr-laerm/emissionsstandards/binnenschiffe#energieverbrauch-und-emissionen-von-binnenschiffen
Die niedrigen Schadstoffgrenzwerte für Binnenschiffe sorgen für eine 
verminderte Luftqualität entlang der Wasserstraßen und besonders in den 
städtischen Häfen. 
Um die negativen gesundheitlichen Auswirkungen auf Menschen zu verringern, muss der Schadstoffausstoß der Binnenschiffe reduziert werden.
Die meisten Binnenschiffe auf dem Rhein werden mit Diesel angetrieben. 
Binnenschiffe die Wasserstoff als Hauptenergiequelle verwenden, könnten Emissionsfrei betrieben werden. Speziell die Rheinschiffe könnten zusätzlich von der im Rheinland ansässigen Chemieindustrie profitieren. Denn in 
vielen der chemischen Prozesse entsteht Wasserstoff als Nebenprodukt und könnte 
hier direkt betankt und verwendet werden, um die Chemieindustrie mit neuen 
Rohstoffen zu beliefern.

	\chapter{Grundlagen}\label{ch:grundlagen}

	\chapter{Methodisches Vorgehen}\label{ch:methodisches_vorgehen}

	%\chapter{Simulationsmodel}
Vorstellung des Models -> Berechnung nach Grundlagen Kap.

Vessel L =110 m - B = 11,4 m - T =2,8 m - \(\Delta\)= 3 140 T - Deadweight: 
2300 T, 
Lightweight: 840 T - Installed power : 1000 kW
Lwl=109,240m Cb=0,8942 Cm=0,9980 Cp=0,8960 Cwp=0,9387 Lcb\%=0,745 App=27,31m2 
(1+k2)=1,50 \(v_{max}\) = 15,2 km/h

Nozzle 19A KA 4-70 (2) 2 Diam 1,59 m 4 blades p/D=1,190 Ae/Ao=0,700    -   
Sm=1707m2 At=0,00m2 Abt=0,00m2 Hb=0,00m2

Waterway : W =156,00 m   w =120,00 m    h=4,50 m Vcr = 16,4km/h 
\(\sqrt{gh}\)23,9km/h 93\%Vcr = 15,2km/h

Hm=3,98m   Bc=138,0m       Bi=198<->225m   Ac=621,0m2   Ab=31,9m2   
Bc/B=12,11   h/T=1,61   m=Ab/Ac=0,051   Ac/Ab=19,5

\begin{table}[htbp]
	\caption{Vessel dimensions -> 	Dand \& Ferguson, Romisch (wenn B/T 
	abgerundet)}
	\centering
	\begin{tabular}{c c c c c }
		& CB &B/T&L/B&L/T \\
		Beispiel Vessel & 0,8942 & 4,0714 & 9,6491 & 39,2857 \\
	
	\end{tabular}
\end{table}
\begin{table}[htbp]
	\caption{Waterway \& interface vessel - Waterway -> Romisch, Huuska-Guliev, 
	Yoshimura- Ohtsu, Eryuzlu 2}
	\centering
	\begin{tabular}{c c p{3cm} c c c }
		& h/T &\(A_B/A_c\)&\(h_T/h\) &L/h & Bc/B(a) \\
		Beispiel Vessel & 1.868 & (B*h)/(h*(W+w)/2) = 31.92 /628.6 = 0.0508& -  
		& 39.28 & Ac/h / B = 19,69\\
		
		
	\end{tabular}
\end{table}
	
	\chapter{Ergebnisse}\label{ch:ergebnisse}
Hier werden deine erschreckenden Erkenntnisse präsentiert.

\begin{figure}[htbp]
	\centering
	\includegraphics[width=0.9\textwidth]{figures/stripes_GLOBE---1850-2019-MO-withlabels}
	\caption[Titel der Figure]{Beschreibungstext Bla bla bla viel beschreiben sehr gut. \cite{Hawkins.2019}}
	\label{fig:DieLableIhAuhhNoo}
\end{figure}

	\input{Kapitel/Diskussion.tex}
	\chapter{Fazit und Ausblick}\label{ch:fazit_ausblick}

	
	\printbibliography
	\addcontentsline{toc}{chapter}{Literatur}
	
	
	\newpage
	\addcontentsline{toc}{chapter}{Anhang}
	\appendix
	\pagenumbering{gobble}
	\chapter*{Anhang}\label{ch:Anhang}
\chaptermark{Anhang}
%\addcontentsline{toc}{chapter}{Anhang}

\setcounter{figure}{0}
\renewcommand{\thefigure}{\Alph{section}.\arabic{figure}}
\renewcommand{\thesection}{\Alph{section}} 
\setcounter{table}{0}
\renewcommand\thetable{\Alph{section}.\arabic{table}}


\section{Warming Stripes}
\begin{figure}[htbp]
	\centering
	\includegraphics[width=\textwidth]{anhang/_stripes_EUROPE-Germany-Nordrhein_Westfalen-1881-2019-DW-withlabels.png}
	\caption[Titel der Figure]{Beschreibungstext Bla bla bla viel beschreiben sehr gut. \cite{Hawkins.2019}}
	\label{fig:DieLableIhAuhhNooo}
\end{figure}

	
	
\end{document}