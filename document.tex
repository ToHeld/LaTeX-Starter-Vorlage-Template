\documentclass[11pt,a4paper]{scrreprt}
%\pdfminorversion=7
\usepackage[T1]{fontenc}
\usepackage[ngerman]{babel}
\usepackage{scrhack}
\usepackage{amsmath}
\usepackage{amsfonts}
\usepackage{amssymb}
\usepackage[]{siunitx}  %% For proper units (e.g. \si{kg.m.s^{-1}}) and numbers (e.g. \num{.3e45})
%%(e.g. \num{.3e45})
\sisetup{output-decimal-marker = {,}}
\usepackage{icomma} %% entfernt das Leerzeichen hinter einem Komma in der 
%%Matheumgebung bei 3,141, aber nicht bei f(x, y)
\usepackage[pdftex]{graphicx}
\usepackage[colorinlistoftodos]{todonotes} % make notes in doc e.g. \todo{Make a cake \ldots}, \missingfigure
\usepackage{epstopdf}
\usepackage[hidelinks]{hyperref}
\usepackage[numbered]{bookmark} %% Numbers in PDF table of contens
\usepackage{float}
\usepackage{placeins}%% Ergmöglich \FloatBarrier
\usepackage[labelfont=bf, figurename= Abb., tablename=Tab.]{caption} %Naming is optional
\captionsetup[table]{justification=justified,singlelinecheck=false}

%% citation
\usepackage{url}  %% For citing webpages
%\usepackage[style=ieee,backend=biber]{biblatex}	%% Meistgenutzter Zitierstil im Ingenieurswesen
\usepackage{csquotes} %% \enquote{Text} \foreignquote{Sprache}{Text}
\usepackage[backend=biber, %% Hilfsprogramm "biber" (statt "biblatex" oder "bibtex")
style=numeric, %% Zitierstil (siehe Dokumentation)
sorting=none, %% none => choronoligisch
natbib=true, %% Bereitstellen von natbib-kompatiblen Zitierkommandos
hyperref=true, %% hyperref-Paket verwenden, um Links zu erstellen
]{biblatex}
\addbibresource{kapitel/x_literaturverzeichnis.bib}
\usepackage{acronym}
\usepackage{tabularray}
\UseTblrLibrary{booktabs} 
\usepackage{booktabs}
\usepackage{multirow}
\usepackage{hhline}
\usepackage{subcaption}
\usepackage{multicol} % Multi Columns für enumerations
%\usepackage{showframe} %% Debuging Helpfull
%%%%%%%%%%%%%%%%%% Document Style
\hfuzz=2pt %% Warnung für Bad Box erst ab 2pt
\usepackage{lmodern}		%% Spezial Font
\usepackage[onehalfspacing]{setspace}	%% Abstand zwischen den Zeilen
\usepackage[headsepline=true]{scrlayer-scrpage}
\pagestyle{scrheadings}
\lohead[]{\leftmark}	% bei odd page \lohead,\cohead und \rohead
\cohead[]{}				% bei even page \lehead,\cehead und \rehead
%\rohead[]{\rightmark} 	% bei zweiseitig \ihead, \chead \ohead
\rohead[]{{\small Super coole \LaTeX Vorlage}} 				%	HIER DEN HEADER AENDERN
\automark[section]{chapter}
%%%%%%%%%%%%%%%%% Hurenkind und Schusterjunge -> https://bfy.tw/P79K
\clubpenalty=10000
\widowpenalty=10000 
\displaywidowpenalty=10000
%%%%%%%%%%%%%%%%%
%\usepackage{blindtext}
\newcommand{\Author}{Merle Mustermensch}
\newcommand{\studentNumber}{123456}
\newcommand{\addressStreet}{Wunschaddresse 42}
\newcommand{\addressZipCode}{1337 Stadt}
\newcommand{\email}{Merle.Mustermensch@h-brs.de}
\newcommand{\Title}{My new created Title in multiple lines but it takes some words}
\newcommand{\Keywords}{H-BRS, Industrie 4.0, Energiewende, TanjaUltras}
\newcommand{\ExaminerOne}{Prof. Dr. Super Professorin}
\newcommand{\ExaminerTwo}{Prof. Dr. Mega Professor}
\newcommand{\submissionDate}{\today} % 12.03.2021

\hypersetup{
pdftitle={\Title}, %
pdfauthor={\Author}, %
pdfkeywords={\Keywords}}

\begin{document}
	\pagenumbering{gobble} 
	\begin{titlepage}
	%\setlength{\parindent}{0pt}%Einrückung auf Titelseite verhindern
	\begin{figure}[htbp]
		\includegraphics[height=1.4cm]{Kapitel/xx_Logo_HBRS_74mm_Pfade-eps-converted-to.pdf}
	%	\hfill
	%	
	%\includegraphics[height=1.4cm]{Kapitel/xx_BRS-Blau_Schwarz_Ohne_Hintergrund_HD.png}
	\end{figure}
 % \includegraphics[width=8cm]{Logo_HBRS_74mm_Pfade.eps}
  \linespread{1.4}%\renewcommand{\baselinestretch}{1.4}\normalsize
  \vspace{2cm}
  \begin{center}

%% einen Typ auswählen
    \begin{Huge}\textbf{Thesis}\end{Huge} \\
    \vspace{1.5cm}
%% einen Studiengang auswählen
    \begin{Large}{Master Maschinenbau - Virtuelle Produktentwicklung}\end{Large} \\
   % \begin{large}
   % 	Forschungsprojekt
   % \end{large}

    \vspace{0.7cm}
    \linespread{1.2}%\renewcommand{\baselinestretch}{1.2}\normalsize
    \begin{huge}
      \textbf{\Title}
    \end{huge}
    \linespread{1.5}%\renewcommand{\baselinestretch}{1.5}
    \normalsize
    \vspace{1cm}%{0.7cm}

    \begin{Large}{\Author \\}
    \end{Large}

    \begin{Large}% Fachbereich
        Fachbereich Elektrotechnik, Maschinenbau \\ und Technikjournalismus
    \end{Large}
  \end{center}

%  \vfill
\vspace*{\fill}

  \begin{large}
    {
      \begin{tabular}{ll}
      Name: & \Author \\
      Matrikel-Nr.: & \studentNumber \\
      Adresse: & \addressStreet \\
       & \addressZipCode \\
      Mail: & \email \\
      Erstprüferin:  & \ExaminerOne \\
      Zweitprüfer: & \ExaminerTwo \\
      Eingereicht am: & \submissionDate
      
      \end{tabular}
	}
  \end{large}

\end{titlepage}

	%\pagestyle{empty}
\section*{Erklärung}

\Author \\
\addressStreet\\
\addressZipCode \\

„Ich versichere hiermit, die von mir vorgelegte Arbeit selbstständig verfasst zu haben. Alle Stellen, die wörtlich oder sinngemäß aus veröffentlichten oder nicht veröffentlichten Arbeiten anderer entnommen sind, habe ich als entnommen kenntlich gemacht. Sämtliche Quellen und Hilfsmittel, die ich für die Arbeit benutzt habe, sind angegeben. Die Arbeit hat mit gleichem Inhalt bzw. in wesentlichen Teilen noch keiner anderen Prüfungsbehörde vorgelegen.

Mir ist bewusst, dass sich die Hochschule vorbehält, meine Arbeit auf plagiierte Inhalte hin zu überprüfen und dass das Auffinden von plagiierten Inhalten zur Nichtigkeit der Arbeit, zur Aberkennung des Abschlusses und zur Exmatrikulation führen können.“
\vspace{3cm}


\noindent\parbox[t]{5cm}{\underline{\hspace{5cm}}\\\noindent Ort, Datum}%
\hfill%
\noindent\parbox[t]{5cm}{\noindent\underline{\hspace{5cm}}\\\noindent \Author}%

	\pagenumbering{Roman}
	\tableofcontents
	\clearpage
	\listoffigures%% Abbildungsverzeichnis || List of Figures
	\addcontentsline{toc}{chapter}{\listfigurename}
	\clearpage %\cleardoublepage %for openright
	\listoftables 
	\addcontentsline{toc}{chapter}{\listtablename}
	\clearpage %\cleardoublepage %for openright
	\addcontentsline{toc}{chapter}{Todo List}
	\listoftodos[Meine Todo list]\todo{Liste vor Druck abarbeiten und löschen}
	\input{kapitel/Acronym.tex}
	\newpage
	\pagenumbering{arabic}
	\pagestyle{scrheadings}
	
	\input{kapitel/Einleitung.tex}
	\chapter{Grundlagen}\label{ch:grundlagen}
In der \texttt{00\_Beispiel.txt} sind tolle \LaTeX Beispiele, damit man nicht immer wieder googeln muss.

\section{RAS-Syndrom}\label{sec:ras-syndrom}
Akronyme sollten mit dem Befehl \texttt{$\backslash$ac} benutzt werden damit eine Verlinkung zum Verzeichnis erstellt wird, wie hier \ac{LCD} und \ac{PIN}. Viele Menschen nennen sie auch \ac{LCD}-Display oder \ac{PIN}-Nummer. Das ist RAS-Syndrom also ein \ac{RAS}-Syndrom und \ac{RAS}-Syndrome sollten vermieden werden.

\section{Stil}\label{sec:stil}
Wer ein Unterkapitel\todo{Das Stil Kapitel fertig schreiben} erzeugt, sollte immer mindestens ein zweites erzeugen.
	\chapter{Methodisches Vorgehen}\label{ch:methodisches_vorgehen}
Man kann Tabellen unterschiedlich schön gestalten. Ein Beispiel:
% First version of table.
\begin{table}[hbtp]
	\centering
	\caption[Unschöne Tabelle]{Diese Tabelle ist nicht schön. Sie hat vertikale Linien, die das Gesamtbild unruhig wirken lassen und enthält redundante Information.}
	\label{tab:unschoen}
	\begin{tabular}{|l|c|c|c|c|c|c|l|}\hline
		&& \multicolumn{3}{c|}{\(\text{tol}= u_{\text{single}}\)} & \multicolumn{3}{c|}{\(\text{tol}= u_{\text{double}}\)}\\
		&Test& $mv$  & Rel.~err & Time    & $mv$  & Rel.~err & Time   \\\hline
		\texttt{trigmv}   & Typ 1 & 11034 & 1.3e-7 & 3.9 & 15846 & 2.7e-11 & 5.6 \\
		\texttt{trigexpmv}& Typ 1  & 21952 & 1.3e-7 & 6.2 & 31516 & 2.7e-11 & 8.8 \\
		\texttt{trigblock}& Typ 2  & 15883 & 5.2e-8 & 7.1 & 32023 & 1.1e-11 & 1.4e1\\
		\texttt{expleja}  & Typ 2  & 11180 & 8.0e-9 & 4.3 & 17348 & 1.5e-11 & 6.6 \\\hline
	\end{tabular}
\end{table}


% Second version of table, with booktabs.
\begin{table}[hbtp]
	\centering
	\caption[Schöne Tabelle]{Diese Tabelle wirkt gleich viel aufgeräumter und ist durch die Vermeidung redundanter Information auch schneller verständlich.}
	\label{tab:schoen}
	\begin{tabular}{lccccccl}\hline
	&& \multicolumn{3}{c}{\(\text{tol}= u_{\text{single}}\)} & \multicolumn{3}{c}{\(\text{tol}= u_{\text{double}}\)} 
	\\\cmidrule(lr){3-5}\cmidrule(lr){6-8}
	&Test & $mv$  & Rel.~err & Time    & $mv$  & Rel.~err & Time\\\midrule
	\texttt{trigmv}   &\multirow{2}{*}{Typ 1} & 11034 & 1.3e-7 & 3.9 & 15846 & 2.7e-11 & 5.6 \\
	\texttt{trigexpmv}& & 21952 & 1.3e-7 & 6.2 & 31516 & 2.7e-11 & 8.8 \\
	\texttt{trigblock}&\multirow{2}{*}{Typ 2} & 15883 & 5.2e-8 & 7.1 & 32023 & 1.1e-11 & 1.4e1\\
	\texttt{expleja}  & & 11180 & 8.0e-9 & 4.3 & 17348 & 1.5e-11 & 6.6 \\\bottomrule
	\end{tabular}
\end{table}
	\chapter{Durchführung}\label{ch:durchfuehrung}%keine Umlaute benutzen!
\missingfigure{Hier könnte Ihre Werbung stehen}

Text\todo[color=green!40]{Umbedingt eine tolle Grafik erstellen}
	
	\chapter{Ergebnisse}\label{ch:ergebnisse}
Hier werden deine erschreckenden Erkenntnisse präsentiert.

\begin{figure}[htbp]
	\centering
	\includegraphics[width=0.9\textwidth]{figures/stripes_GLOBE---1850-2019-MO-withlabels}
	\caption[Titel der Figure]{Beschreibungstext Bla bla bla viel beschreiben sehr gut. \cite{Hawkins.2019}}
	\label{fig:DieLableIhAuhhNoo}
\end{figure}

	\input{kapitel/Diskussion.tex}
	\chapter{Fazit und Ausblick}\label{ch:fazit_ausblick}

Es gibt noch eine Menge Todos.
\listoftodos[Meine Todo list]
	
	\printbibliography
	\addcontentsline{toc}{chapter}{Literatur}
	
	
	\newpage
	\appendix
	%\pagenumbering{gobble}
	\input{kapitel/Anhang.tex}
	
	
\end{document}
