\chapter{Einleitung}\label{ch:Einleitung}
Die Ladung eines Binnenschiffs entlastet den Straßenverkehr um 40 bis 180 LKW. 
%{2019-12-02_STG2018-Dahlke-Friedhoff_Energietraeger_fuer_die_Binnenschifffahrt_von_Morgen}
Im Vergleich zu einem 40-Tonnen-Sattelzug verbraucht ein Schiff weniger 
als ein Drittel an Energie und stößt entsprechend weniger Treibhausgase aus. 
In der Luftschadstoffbilanz ist der Transport per Binnenschiff jedoch 
schlechter als per Lkw.\cite{UmweltBundesamt.2020}
%https://www.umweltbundesamt.de/themen/verkehr-laerm/emissionsstandards/binnenschiffe#energieverbrauch-und-emissionen-von-binnenschiffen
Die niedrigen Schadstoffgrenzwerte für Binnenschiffe sorgen für eine 
verminderte Luftqualität entlang der Wasserstraßen und besonders in den 
städtischen Häfen. 
Um die negativen gesundheitlichen Auswirkungen auf Menschen zu verringern, muss der Schadstoffausstoß der Binnenschiffe reduziert werden.
Die meisten Binnenschiffe auf dem Rhein werden mit Diesel angetrieben. 
Binnenschiffe die Wasserstoff als Hauptenergiequelle verwenden, könnten Emissionsfrei betrieben werden. Speziell die Rheinschiffe könnten zusätzlich von der im Rheinland ansässigen Chemieindustrie profitieren. Denn in 
vielen der chemischen Prozesse entsteht Wasserstoff als Nebenprodukt und könnte 
hier direkt betankt und verwendet werden, um die Chemieindustrie mit neuen 
Rohstoffen zu beliefern.
