\newpage
\chapter*{Abkürzungsverzeichnis}
\begin{acronym}
%	\acro{hm}[H\&M]{Holtrop \& Mennen}
\end{acronym}

\newpage
% Hier evtl: https://www.ctan.org/pkg/glossaries
\chapter*{Variablenverzeichnis}
\begin{table}[ht]
	\begin{tabular}{rl}
	
	\(\beta\) & Trimm: \(m\)\\		
	\(\rho\)& Wasserdichte: \(kg\cdot m^{-3}\)	\\		
	&	\\
	\(A_b\)& Blockfläche Schiff: \(m^2\)\\
	\(B_c\)& Normierte Breite: \(m\)\\
	\(Fn\) & Froude-Zahl: \(-\)\\
	\(Fn_h\) & Froude'sche Tiefenzahl: \(-\)\\	
	\(g\) & Erdbeschleunigung: \(m \cdot s^{-2}\)\\
	\(h\)& Wassertiefe: \(m\)\\
	\(H_m\)& Normierte Wassertiefe: \(m\)\\
	\(lcb\) & Schwerpunkt der Verdrängung: \(\%\)\\
	\(P_b\)& Benässte Schiffsfläche im Querschnitt: \(m\)\\
	\(P_c\)& Benässte Flussfläche im Querschnitt: \(m\)\\
	\(R\) & Widerstand: \(kN\)\\
	\(R_h\)& Hydraulischer Radius Schiff-Wasserstraße: \(m\)\\		
	
	\(u\) & Rückströmung: \(m\cdot s^{-1}\)\\
	\(W\)& Breite Wasseroberfläche: \(m\)\\
	\(w\)& Breite Flussbett: \(m\)\\
	\(S_d\) & Squat: \(m\)\\	
 	\(z\) & Absenkung des Wasserstands: \(m\)\\
\end{tabular}
\end{table}